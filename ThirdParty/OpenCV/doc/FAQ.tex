\chapter{FAQ}
\section{First Section}
\subsection{Initialization}

\subsubsection*{CreateImage}
\addcontentsline{toc}{subsubsection}{CreateImage} 

Creates header and \textsf{allocates} data

\begin{shaded}
\begin{verbatim}
IplImage* cvCreateImage( CvSize size,
                         int depth,
                         int channels );
\end{verbatim}
\end{shaded}

\begin{description}
\item[\texttt{size}] Image width and height
\item[\texttt{depth}] Bit depth of image elements.  Can be one of:
\begin{description}
\item[IPL\_DEPTH\_8U] unsigned 8-bit integers
\item[IPL\_DEPTH\_8S] signed 8-bit integers
\item[IPL\_DEPTH\_16U] unsigned 16-bit integers
\item[IPL\_DEPTH\_16S] signed 16-bit integers
\item[IPL\_DEPTH\_32S] signed 32-bit integers
\item[IPL\_DEPTH\_32F] single precision floating-point numbers
\item[IPL\_DEPTH\_64F] double precision floating-point numbers
\end{description}
\item[\texttt{channels}] Number of channels per element(pixel). Can be 1, 2, 3 or 4. The channels are interleaved, for example the usual data layout of a color image is:
\begin{lstlisting}
b0 g0 r0 b1 g1 r1 ...
\end{lstlisting}
Although in general IPL image format can store non-interleaved images as well and some of OpenCV can process it, this function can create interleaved images only.

\end{description}

The function cvCreateImage creates the header and allocates data as in the method of~\cite{author_conf_year}.  This call is a shortened form of 
\begin{lstlisting}
header = cvCreateImageHeader(size,depth,channels);
cvCreateData(header);
\end{lstlisting}
